## Input for the code

# Name file (between ' with spaces on both sides and no spaces in name)
' Wedge_optimization_PLIC_many '

## PARAMETERS

# Density air [kg/m^3] | Density water [kg/m^3] | Dynamic viscosity air [kg/ms] | Dynamic viscosity water [kg/ms]
1.334 1e-10 1e-10

# Gravitational constant [x, y]
0 -9.81

# Surface tension coefficient | Atmospheric pressure [N/m^2] | Gamma air [-]	| Speed of sound water [m/s]
0.0 	1e5 	1.4		1500

# Position of definition of atmospheric pressure [pos, x_0, x_1]
# Atmospheric opens the wall automatic
# pos: Top (1)	| Bottom (3)	| Right (2)		| Left (4) | (0) if none
# Distance x_0 and x_1
1 0 1.0

#-------------------
## DOMAIN

# Domain dimension [x, z] [m]
1.0 0.8

# Grid size and geometrical progression ratio [Nx, Nz, Rx, Rz]
# SIMPLE [Nx, Nz, Rx, Rz] (amount of cells and ratio between smallest and largest cell)
# DETAIL [xL, xR, N, sL, sR] (2x) (a band of N cells between xL and xR, stretch outward with sL and sR )
DETAIL 0.3 0.7 40 1.05 1.05 0.3 0.6 30 1.05 1.05
# SIMPLE 102 82 1 1

#-------------------
## CASE
# For wave no body can be applied yet !
# Free flow (0)	| Wave (1)
0

## WAVE
# Amplitude [m] | Height [m]	| Wave length [m]	| Phase *pi
0.05 	0.50		1		0

# Cosine (1)	| Sine (0)
1

# Direction filled: north (1)	| east (2)	|	south (3)	| west (4)
3

## FREE FLOW
# Way of filling
# [1] Circle        [x_center, z_center, radius]
# [2] Wedge         [x_left, x_right, z_entry, a_degree]
# [3] BLOCK 		[x_left, x_right, z_bottom, z_top, radius]
# [4] WEDGE_BLOCK   [x_left, x_right, z_entry, a_degree, h_block]
# [5] HEIGHT_z      [x_height]
# [6] HEIGHT_x      [z_height]
# [7] CIRCLE_BLOCK  [x_center, z_center, radius, h_block]
# [8] FULL
# [9] PICTURE		[picture relative address, x_centre, y_centre, scale ( standard: 1 pxl = 1 mm)]

# PART 1:
# Filling dimension of water [air, way, x_begin, x_end, z_begin, z_end] [m]
0 0 0 0 0

# PART 2 (Using Geometry.py module):
# Filling dimension of water [air, way, x_begin, x_end, z_begin, z_end] [m]
0 NONE

# PART 3 (Using Geometry.py module):
# Filling dimension of water [air, way, x_begin, x_end, z_begin, z_end] [m]
0 HEIGHT_Z 0.495

#-------------------
## BODY DEFINITION (single body for now)
# Input per body:
# - dimensions      [way of filling]
# - velocity        [initial velocity (u0, v0), external acceleration (ax, ay)]
# - mass            [mass/density (0/1) + value (m/rho), center of gravity (xg, yg)]
# - motion          [coupled (0) + in direction (cx, cy), around reference (1) + position (px, py)]

# BODY 1:
# Dimensions of body [way, param1, param2, param3, ...] [m]
# picture is 800px wide, scale 0.25 aligns with B = 0.2m, H = 0.1m, ch = 0.05m
picture NURBS.png 0.50 0.59 1

# BODY 2:
# Dimensions of body [way, param1, param2, param3, ...] [m]
NONE

# Initial velocity of body [u0, v0, ax, ay]
0 -4.0 0 0

# Mass of body [m/rho, value, xg, yg]
1 500 0 0

# Motion of body [uncoupled: 0 0 0, coupled: 0 1 1, pendulum: 1 px py cyl_radius length I C K start_rad]
0 0 1

#-------------------
## INFLOW OF WATER
# Types of inflow:
# [0] Closed                [-]
# [1] Inflow                [velocity (positive = inflow) (m/s), x_begin (m), x_end (m)]
# [2] Wave in (LEFT/RIGHT)  [mean surface level (m)] (reads waveparam.csv)
# [3] Periodic (L/R)        [-] (Set at left and right!)
# [4] Open boundary         [x_begin(m), x_end(m)]
#     by con. pressure ext.

# Start time [s] | End time [s] | Ramp time [s] (Only works for steady inflow)
0 0 0

# Top wall
# Inflow type | [param1, param2, ...]
0

# Right wall
# Inflow type | [param1, param2, ...]
0

# Bottom wall
# Inflow type | [param1, param2, ...]
0

# Left wall
# Inflow type | [param1, param2, ...]
0

#-------------------
## TIME
# Time duration [s]	| Time step initial [s]		| Time step min [s]	| Time step max [s] | Iter initial condition
0.05 0.00001 0.0000001 0.001 0

# CFL_max | Eps_max error | Omega_0 relaxation Aitken
0.25		1e-10	        1

# CFL crit
# - Only inertial (1)
# - Inertial + capillary + viscous (0)
1

#-------------------

## SIMULATION TYPE
# Second order upwind (1)	| First order upwind (0) (second-order does not work for one-phase)
0

# Adam-Bashforth (1)	| Forward Euler (0) (both for the convective as well as diffusive term)
0

# Height function (1)	| Height function original (2) | None (0)
0

# Quarter-cell based (2) | Gravity-consistent interpolation (1)	|	Central-weighted averaging (0) (not needed for one-phase)
2

# PLIC Youngs (1)	| SLIC Hirt Nichols (0) | PLIC HF (2) | PLIC HF/Youngs (3) / Coeff (values/ALL = -1) (BODY AND FLUID)
1 0

# Stamping (1)	| None (0)
0

# Incompressible (0)	| Air compressible (1)	| Compressible (2) | One-phase (3) (AERATION: 2)
0

# Correcting divergence term transport (2) | Correcting divergence term transport with pressure (1) | None (0) (AERATION: 1)
0

# Scheme fluid flux: Macho scheme (1)	|  Cosmic (0) | EMFPA (2) | Unsplit (3)
0

# No-slip at wall (-1)	| Free-slip (1)
1

# Weighted (1) | Baraldi (0)
0

# Labelling Kleefsman (1) | Labelling alternative (0) (Recommended in case of using Body (Same labelling))
0

# SE velocities constant extrapolation (1) | divergence free (0)
1

# Conservative (1) (CFL < 0.5) | Non-conservative (0)
1

# Scheme body flux: Macho scheme (1) (only when same labelling) | Cosmic (0) (only when same labelling) | EMFPA (2) | Unsplit (3)
3

# Body coupling: [none (0), 1way (1), 2way (2)] | Time integration of force: [BEuler (0), Crank Nicolson (1)]
2 0

# Curvature based on 5x5 (1) | 3x3 (0)
0

# Cell merging: yes(1) no(0) | Value
0 0.05
 
# Angle body [deg] (0 works)
90.0

# Filter sharp corners: yes (1) no (0) | size of filter (default = 8)
0 0

#-------------------

## INITIAL CONDITIONS
# Load initial conditions for (1 [xlsx] / 2 [csv]) or default (0): (fraction overwrites other fraction inputs)
# Pressure 		| Velocity		| Density		| Fraction
0   0   0   0

# Pressure (yes/no (1/0), value, x_min, x_max, y_min, y_max)
0 0 0 0 0 0

# Velocity (yes/no (1/0), value_x, value_y)
0 1 1

#-------------------

## AERATION INITIAL
# Initial volume fraction (AERATION LEVEL IN FRACTION)
0.00

#-------------------

## POSTPROCESSING
# dir = 1 when you want to measure on position x in height y
# dir = 0 when you want to measure on position y in height x
# When value of x/y is negative, measuring from another direction
# Measurement of water: yes (1)/no (0), (x/y, dir) h1, (x/y, dir) h2, (x/y, dir) h3, (x/y, dir) h4 [m]
0   0 1  0.5 1  100.0 1  200.0 1

# Measurement of pressure: yes (1)/no (0), (x, y) p1, (x, y) p2, (x, y) p3, (x, y) p4
0	0.3750 0.0001	0.0001 0.0001	0.7499 0.0001     1.4999 0.0001

# Measurement of body_force and position: yes (1)/no (0), number (1, 2, 3)
# Saved in horizontal, vertical, x, y, speed x, speed y
1 1

# Measurement force on wall
0

# Measurement of mass conservation: yes (1)/no (0) | Energy conservation and momentum: yes (1)/no (0)
0 0

# Save data: yes (1)/no (0), amount of steps,
# [(fake)fraction, velocity, inter_vel, density, pressure, cell, reconstr, fracB]
0	4	1 1 0 0 1 0 0 0

# Save figures: yes (1)/no (0), amount of steps, [fraction, velocity, density, pressure, cell], alpha
# alpha = 0.0: No visible object in plot
# alpha = 1.0: Only visible object in plot
1 	10	1 1 0 1 1	0.3

# Save VTK: yes (1)/no (0), amount of steps[dt], [(fake)fraction, velocity, density, pressure, cell, body]
0	0.0001	1 1 0 1 1 1

# Save pressure profile around object: yes (1)/no (0), amount of steps[dt]
0  0.0001

# Save reconstruction coordinates of PLIC/BLIC: yes (1)/no (0), amount of steps[dt
1 0.0001
